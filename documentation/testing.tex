\section{Тестирование}
	\parindent=0cm
	Во всех тестах скорость обучения $\alpha = 0.05$, количество эпох обучения $75000$.
\subsection{Тест 0}
	В данном тесте обучающая выборка равна контрольной и содержит 50 примеров сломанных мячей и 50 целых.
	ИНС один раз ошиблась в отрицательную сторону и 4 раза в положительную.
	Средняя квадратичная функция ошибки $E = 0.03035$.

	\includegraphics{test0.png}

\newpage
\subsection{Тест 1}

	В данном тесте обучающая выборка содержит 37 примеров сломанных мячей и 37 целых.
	Контрольная выборка состоит из 13 сломанных и 13 целых мячей.
	ИНС один раз ошиблась в отрицательную сторону и 2 раза в положительную.
	Средняя квадратичная функция ошибки $E = 0.05817$.

	\includegraphics{test1.png}

\newpage
\subsection{Тест 2}
	В данном тесте обучающая выборка содержит 13 примеров сломанных мячей и 13 целых.
	Контрольная выборка состоит из 37 сломанных и 37 целых мячей.
	ИНС дважды ошиблась в отрицательную сторону и 4 раза в положительную.
	Средняя квадратичная функция ошибки $E = 0.04221$.

	\includegraphics{test2.png}
	\parindent=0.2cm
