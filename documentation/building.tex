\section{Сборка и использование}
\subsection{Сборка проекта}
	Для сборки конечного приложения необходимо запустить \textbf{make} из корневого каталога проекта.

\subsection{Использование}
	Чтобы подготовить записи для работы программы, перевести звук в предикторы необходимо:
	\begin{enumerate}
		\item Создать в корневом каталоге проекта следующие подкаталоги:
		\begin{itemize}
			\item \tbsh wav\tbsh broken\tbsh\qquad--- для записей звука повреждённых мячей;
			\item \tbsh wav\tbsh whole\tbsh\qquad--- для записей звука целых мячей;
		\end{itemize}
		\item Запустить \textbf{make} с параметром \textbf{pred}.
	\end{enumerate}
	В корневом каталоге будет создан файл \textbf{pred*.p} --- предикторы для заданного набора записей.
    * --- время создания файла.

	Для использования программы:
	\begin{enumerate}
		\item Запустить \textbf{main} из корневого каталога;
		\item Написать имя входного файла, содержащего \textbf{обучающую} выборку;
		программа выведет результат обучения ИНС --- предсказания классификации примеров из обучающей выборки;
		\item Написать имя входного файла, содержащего \textbf{контрольную} выборку;
		программа выведет предсказания классификации примеров из контрольной выборки.
	\end{enumerate}
	В [ ] написаны правильные ответы; после вывода предсказаний также будет показано значение функции ошибки.