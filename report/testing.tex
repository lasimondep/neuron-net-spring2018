\section{Тестирование}
	Во всех тестах в нейросети использовалось два скрытых слоя, количество эпох обучения: 50000.
\subsection{Тест 0}
	\parindent=0cm
	В данном тесте обучающая выборка равна контрольной и содержит все 100 примеров из исходных данных.
	Наилучший результат получается при конфигурации сети по 10 нейронов на каждом скрытом слое,
	скорость обучения $\alpha = 0.05$.

	Ошибочных предсказаний для контрольной выборки 6\%.
	\parskip=0cm

	\includegraphics[width=450px]{test0.png}
	\parskip=0.2cm


\newpage
\subsection{Тест 1}
	В данном тесте обучающая выборка содержит 13 примеров сломанных мячей и 13 целых.
	Контрольная выборка состоит из 37 сломанных и 37 целых мячей.
	Наилучший результат получается при конфигурации сети по 10 нейронов на каждом скрытом слое,
	скорость обучения $\alpha = 0.05$.

	Ошибочных предсказаний для контрольной выборки 8.11\%.
	\parskip=0cm

	\includegraphics[width=380px]{test1.png}
	\parskip=0.2cm

\newpage
\subsection{Тест 2}
	В данном тесте обучающая выборка содержит 20 примеров сломанных мячей и 10 целых.
	Контрольная выборка состоит из 30 сломанных и 40 целых мячей.
	Наилучший результат получается при конфигурации сети: 15 нейронов на первом и 10 нейронов
	на втором скрытом слоях, скорость обучения $\alpha = 0.08$.

	Ошибочных предсказаний для контрольной выборки 10\%.
	\parskip=0cm

	\includegraphics[width=380px]{test2.png}
	\parskip=0.2cm

\subsection{Тест 3}
	В данном тесте обучающая выборка содержит 10 примеров сломанных мячей и 20 целых.
	Контрольная выборка состоит из 40 сломанных и 30 целых мячей.
	Наилучший результат получается при конфигурации сети по 10 нейронов на каждом скрытом слое,
	скорость обучения $\alpha = 0.05$.

	Ошибочных предсказаний для контрольной выборки 8.57\%.
	\parskip=0cm

	\includegraphics[width=380px]{test3.png}
	\parskip=0.2cm