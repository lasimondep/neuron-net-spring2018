\section{Реализация}
\subsection{Подготовка предикторов}
	Для подготовки предикторов из записей звука используется библиотека работы с wave-файлами и быстрое преобразование
	Фурье из библиотеки \textbf{numpy}, реализованные в Python~3.

	Весь исходный код для подготовки предикторов находится в подкаталоге \textbf{./predictors/}.

	В файле \textbf{wavetransform.py} написан код, который последовательно обрабатывает файлы из подкаталога с
	записями звуков	\textbf{./wav/} следующим образом:
	\begin{enumerate}
		\item Из файла считывается массив сэмплов;
		\item Данный массив преобразуется при помощи функции \texttt{numpy.rfft} в спектр;
		\item Во временные промежуточные файлы (для сломанных и целых мячей отдельно) записывается промежуточный результат:
				сначала интенсивности, как абсолютные значения из спектра, затем~--- частоты, которые соответствуют данному распределению.
	\end{enumerate}

	После этого работает код, написанный в файле \textbf{predictors.cpp}:
	\begin{enumerate}
		\item Каждые интенсивность и частоты связываются в единый массив пар, в которых на первом месте стоит
				значение интенсивности, на втором~--- соответствующая частота \texttt{std::pair<double, double>};
		\item Полученный массив сортируется по неубыванию интенсивностей в спектре;
		\item Из отсортированного массива выводятся 20 частот с конца, для каждого примера выводятся значения правильных
				ответов на него (1~--- для целых, 0~--- для сломанных мячей).
	\end{enumerate}

	Выходной файл с предикторами создаётся в корневом каталоге проекта. Его имя генерируется по
	времени запуска генерации и имеет вид \textbf{pred*.p}.

\subsection{Нейросеть}
	Исходный код основного приложения находится в подкаталоге \textbf{./src/}.

	Подготовленные данные считываются из файла с предикторами в функции \texttt{reading} (реализация в \textbf{body.cpp}) в массив
	векторов \texttt{input}, правильные ответы в вектор \texttt{y}.

	Нейросеть реализована в виде класса \texttt{NeuronNet}, в котором создаются вектора, соответствующие слоям нейросети,
	содержащим отдельные нейроны, а также вектора значений активаций нейронов каждого слоя. Предсказание
	нейросети на конкретном примере производится вызовом метода \texttt{NeuronNet::forward\_pass}.

	В нейроне, реализованном в классе \texttt{Neuron}, содержится вектор весов для входящих сигналов этого нейрона,
	а также два метода: сумматорная и активационная функции. Исходные значения весов задаются случайно. Активационная функция
	задаётся через конструктор нейросети и хранится как указатель на функцию (по умолчанию~--- сигмоида).

	Метод \texttt{NeuronNet::forward\_pass} получает в качестве параметра вектор значений предикторов одного примера и проводит последовательную
	активацию нейронов по слоям от входного к выходному, после чего возвращает значение активации нейрона в выходном слое.

	Метод \texttt{NeuronNet::backpropagation} реализует алгоритм обратного распространения ошибки.
	Он получает в качестве параметров вектор значений предикторов и правильный ответ для одного примера, а также константу скорости
	обучения, производные активационной функции нейронов и целевой функции ошибки (по умолчанию~--- квадратичная функция ошибки).

	Сначала считается значение ошибки на выходном слое.	Затем ошибка распространяется на предыдущие слои в обратном порядке.
	Для каждого слоя хранится временный вектор ошибок \texttt{wdelt}.

	Обучение нейросети происходит в функции \texttt{learning} (\textbf{body.cpp}).
	В данной функции заданное количество раз (значение параметра \texttt{epoch}) вызывается метод \texttt{NeuronNet::backpropagation},
	с передачей ему случайного примера из входных данных.

	Функция \texttt{check} (\textbf{body.cpp}) выводит предсказания по выборке, передаваемой в качестве параметра, а также возвращает
	количество ошибочных предсказаний в процентах от размера выборки.