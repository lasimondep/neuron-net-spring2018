\section{Постановка задачи}
	Имеется набор данных~--- оцифрованные записи звука удара мячей для настольного тенниса.
	Каждая запись длительностью примерно 3-6 секунд. Всего записей 100 штук.
	Все представлены в виде wave-файлов, пронумерованных от 1 до 100.

	Записи с 1 по 50 соответствуют сломанным мячам. Типов сломанных мячей четыре: мячи с вмятинами с одной стороны,
	мячи с проколами и разрывами с одной стороны, мячи из одной целой полусферы, мячи с вмятинами и проколами с разных сторон.

	Записи с 51 по 100 соответствуют целым мячам.
	
	Все записи были сделаны при ударах о разные типы поверхностей: кафельный пол, металлическая поверхность электрической плиты,
	деревянная поверхность стола, пластиковая поверхность подоконника.

	Необходимо, используя разные компоновки обучающей и контрольной выборки из исходных данных, получить предсказания о классификации
	каждого примера с возможной ошибкой не более 10\% неправильных ответов от размера контрольной выборки.